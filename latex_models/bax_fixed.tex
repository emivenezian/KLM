\documentclass[12pt,a4paper]{article}
\usepackage[utf8]{inputenc}
\usepackage[spanish]{babel}
\usepackage{amsmath}
\usepackage{amsfonts}
\usepackage{amssymb}
\usepackage{graphicx}
\usepackage{geometry}
\usepackage{booktabs}
\usepackage{float}

\geometry{margin=2.5cm}

\title{Modelo BAX Fixed - Optimización de Carga Aérea KLM}
\author{María Emilia Venezian Juricic}
\date{\today}

\begin{document}

\maketitle

\begin{abstract}
\textbf{Resumen Ejecutivo del Modelo}

El modelo BAX Fixed es una variante experimental del modelo Puttaert que fija las posiciones de los ULDs BAX a sus ubicaciones reales del vuelo. Este modelo fue diseñado para evaluar el impacto de las restricciones operacionales rígidas en el rendimiento del optimizador y sirve como caso de estudio de sobre-restricción.

\vspace{0.3cm}
\noindent\textbf{Enfoque:} Puttaert + Posiciones BAX Fijas
\begin{itemize}
    \item \textbf{Base:} Modelo Puttaert completo (1D-BPP + W\&B integrado)
    \item \textbf{Restricción Adicional:} $f_{j,t_{actual}} = 1$ para todo $j \in J_{BAX}$
    \item \textbf{Fuente Datos BAX:} \texttt{j.actual\_position\_bax} de \texttt{LoadLocationsSpotfire.csv}
\end{itemize}

\vspace{0.3cm}
\noindent\textbf{Características Clave:}
\begin{itemize}
    \item Idéntico a Puttaert EXCEPTO por posiciones BAX fijas
    \item 6 objetivos jerárquicos (mismos que Puttaert)
    \item Variables de linealización completas (L1-L7)
    \item Big M dinámico: $M = \max_i w_i$
    \item Feedback loop con 3D packing
\end{itemize}

\vspace{0.3cm}
\noindent\textbf{Restricción Crítica:}
\begin{equation*}
f_{j,t_{actual}} = 1 \quad \forall j \in J_{BAX}, \quad \text{donde } t_{actual} = \texttt{j.actual\_position\_bax}
\end{equation*}

Esta restricción elimina toda flexibilidad en el posicionamiento de ULDs BAX, forzando al modelo a usar exactamente las mismas ubicaciones que en el vuelo real.

\vspace{0.3cm}
\noindent\textbf{Rendimiento:}
\begin{itemize}
    \item Tasa de éxito: 66.8\% (PEOR de todos)
    \item Tasa de infactibilidad: 33.2\% (50\% más que Puttaert)
    \item Ranking: 5/5 modelos
    \item Degradación respecto a Puttaert: -5.3 puntos porcentuales
\end{itemize}

\vspace{0.3cm}
\noindent\textbf{Contexto Histórico:}

Este modelo fue creado como un experimento controlado para medir el impacto de restricciones operacionales rígidas. Los resultados son reveladores: fijar las posiciones BAX aumenta la infactibilidad de 27.9\% a 33.2\%. Esto demuestra que:

\begin{itemize}
    \item Los ULDs BAX (típicamente pesados y en compartimentos delanteros) tienen impacto crítico en el balance longitudinal
    \item La flexibilidad en posicionamiento BAX es esencial para lograr configuraciones factibles
    \item Las decisiones de posicionamiento BAX y asignación de items regulares están fuertemente acopladas
    \item Restricciones aparentemente "razonables" pueden degradar significativamente el rendimiento
\end{itemize}

\textbf{Lección Principal:} Este modelo sirve como advertencia sobre el peligro de sobre-restringir modelos de optimización basándose en prácticas operacionales, incluso cuando esas prácticas parecen lógicas.

\vspace{0.3cm}
\noindent\textbf{Uso Recomendado:} Análisis de sensibilidad y caso de estudio educativo. NO para optimización operacional.

\vspace{0.3cm}
\noindent\textbf{Implementación:} \texttt{BAX\_Fixed.ipynb}

\noindent\textbf{Resultados:} \texttt{Results\_BAX\_Fixed/}
\end{abstract}

\newpage

\section{Introducción}

El modelo BAX Fixed es una variante del modelo de optimización de carga aérea que implementa restricciones fijas para los ULDs de tipo BAX (Bulk Air Cargo). Este modelo combina el problema de empaquetado 1D con el balance de peso en un solo modelo integrado, pero con asignaciones predefinidas para los ULDs BAX.

\section{Conjuntos}

\begin{align}
I &= \{\text{Items de carga}\} \\
J &= \{\text{ULDs disponibles}\} \\
J_{reg} &= \{j \in J : j \text{ es ULD regular (no BAX/BUP/T)}\} \\
J_{BAX} &= \{j \in J : j \text{ es ULD BAX}\} \\
J_{BUP} &= \{j \in J : j \text{ es ULD BUP}\} \\
J_{T} &= \{j \in J : j \text{ es ULD T}\} \\
T &= \{\text{Posiciones de carga en la aeronave}\} \\
T_{C1} &= \{t \in T : t \text{ en compartimento 1}\} \\
T_{C2} &= \{t \in T : t \text{ en compartimento 2}\} \\
T_{C3} &= \{t \in T : t \text{ en compartimento 3}\} \\
T_{C4} &= \{t \in T : t \text{ en compartimento 4}\} \\
T_{left} &= \{t \in T : t \text{ lado izquierdo}\} \\
T_{right} &= \{t \in T : t \text{ lado derecho}\} \\
P &= \{\text{Grupos de prefijos de items}\}
\end{align}

\section{Parámetros}

\subsection{Parámetros de Items}
\begin{align}
w_i &\in \mathbb{R}^+ \quad \forall i \in I \quad \text{(Peso del item } i\text{)} \\
v_i &\in \mathbb{R}^+ \quad \forall i \in I \quad \text{(Volumen del item } i\text{)} \\
COL_i &\in \{0,1\} \quad \forall i \in I \quad \text{(Item } i \text{ es COL)} \\
CRT_i &\in \{0,1\} \quad \forall i \in I \quad \text{(Item } i \text{ es CRT)}
\end{align}

\subsection{Parámetros de ULDs}
\begin{align}
W_j &\in \mathbb{R}^+ \quad \forall j \in J \quad \text{(Capacidad máxima de peso del ULD } j\text{)} \\
V_j &\in \mathbb{R}^+ \quad \forall j \in J \quad \text{(Volumen del ULD } j\text{)} \\
w_j &\in \mathbb{R}^+ \quad \forall j \in J \quad \text{(Peso del ULD } j\text{)} \\
actual\_position_{BAX}(j) &\in T \quad \forall j \in J_{BAX} \quad \text{(Posición actual del ULD BAX } j \text{ desde LoadLocationsSpotfire.csv)}
\end{align}

\subsection{Parámetros de la Aeronave}
\begin{align}
DOI &\in \mathbb{R} \quad \text{(Dry Operating Index)} \\
OEW &\in \mathbb{R}^+ \quad \text{(Operating Empty Weight)} \\
TOF &\in \mathbb{R}^+ \quad \text{(Take-Off Fuel)} \\
TripF &\in \mathbb{R}^+ \quad \text{(Trip Fuel)} \\
C &\in \mathbb{R} \quad \text{(Constante de conversión)} \\
K &\in \mathbb{R} \quad \text{(Constante de referencia)} \\
ZFW &\in \mathbb{R}^+ \quad \text{(Zero Fuel Weight)} \\
reference\_arm &\in \mathbb{R} \quad \text{(Brazo de referencia)} \\
lemac &\in \mathbb{R} \quad \text{(Leading Edge MAC)} \\
mac\_formula &\in \mathbb{R}^+ \quad \text{(Fórmula MAC)}
\end{align}

\subsection{Parámetros de Compartimentos}
\begin{align}
\Delta_{C1} &\in \mathbb{R} \quad \text{(Delta índice compartimento 1)} \\
\Delta_{C2} &\in \mathbb{R} \quad \text{(Delta índice compartimento 2)} \\
\Delta_{C3} &\in \mathbb{R} \quad \text{(Delta índice compartimento 3)} \\
\Delta_{C4} &\in \mathbb{R} \quad \text{(Delta índice compartimento 4)} \\
W_{max}^{C1} &\in \mathbb{R}^+ \quad \text{(Peso máximo compartimento 1)} \\
W_{max}^{C2} &\in \mathbb{R}^+ \quad \text{(Peso máximo compartimento 2)} \\
W_{max}^{C3} &\in \mathbb{R}^+ \quad \text{(Peso máximo compartimento 3)} \\
W_{max}^{C4} &\in \mathbb{R}^+ \quad \text{(Peso máximo compartimento 4)} \\
W_{max}^{C1C2} &\in \mathbb{R}^+ \quad \text{(Peso máximo compartimentos 1+2)} \\
W_{max}^{C3C4} &\in \mathbb{R}^+ \quad \text{(Peso máximo compartimentos 3+4)}
\end{align}

\subsection{Parámetros Globales Compartidos}
\begin{align}
INDEX\_PAX &\in \mathbb{R} \quad \text{(Índice aportado por pasajeros)} \\
fuel\_index &\in \mathbb{R} \quad \text{(Índice aportado por combustible)} \\
MPL &\in \mathbb{R}^+ \quad \text{(Maximum Payload)} \\
W_{total} &\in \mathbb{R}^+ \quad \text{(Peso total de carga asignada)} \\
number\_of\_opened\_uld &\in \mathbb{Z}^+ \quad \text{(Número mínimo de ULDs a abrir en etapa 1)}
\end{align}

\subsection{Parámetros de Control}
\begin{align}
loadfactor &= 0.8 \quad \text{(Factor de carga volumétrica)} \\
threshold\_volume &= \text{umbral\_volumen\_AKE} \times 10^6 \quad \text{(Umbral de volumen para AKE)} \\
min\_load\_factor &= 0.2 \quad \text{(Factor de carga mínimo)} \\
a_{lat}^{TOW} &= 0.5 \quad \text{(Parámetro balance lateral TOW)} \\
b_{lat}^{TOW} &= 0.5 \quad \text{(Parámetro balance lateral TOW)} \\
a_{lat}^{LW} &= 0.5 \quad \text{(Parámetro balance lateral LW)} \\
b_{lat}^{LW} &= 0.5 \quad \text{(Parámetro balance lateral LW)} \\
M &= \max_{i \in I} w_i \quad \text{(Big M se establece dinámicamente como el peso máximo de los items)}
\end{align}

\noindent\textbf{Funciones auxiliares:}
\begin{itemize}
    \item $\mathbf{1}_{AKE}(j)$, $\mathbf{1}_{PMC\_PAG}(j)$ indican si el ULD $j$ es de tipo AKE o PMC/PAG.
    \item $proximity\_score(t)$ cuantifica la penalidad por ubicar BAX en la posición $t$.
    \item $T_{forbidden}(j)$ contiene posiciones prohibidas para el ULD $j$ y $T_{overlap}(t)$ identifica posiciones que se solapan con $t$.
\end{itemize}

\section{Variables de Decisión}

\begin{align}
f_{jt} &\in \{0,1\} \quad \forall j \in J, \forall t \in T \quad \text{(Asignación del ULD } j \text{ a la posición } t\text{)} \\
w_{ijt} &\in \mathbb{R}^+ \quad \forall i \in I, \forall j \in J, \forall t \in T \quad \text{(Peso del item } i \text{ en ULD } j \text{ posición } t\text{)} \\
u_j &\in \{0,1\} \quad \forall j \in J \quad \text{(Activación del ULD } j\text{)} \\
p_{ij} &\in \{0,1\} \quad \forall i \in I, \forall j \in J \quad \text{(Asignación del item } i \text{ al ULD } j\text{)} \\
z_{ijt} &\in \{0,1\} \quad \forall i \in I, \forall j \in J, \forall t \in T \quad \text{(Asignación del item } i \text{ al ULD } j \text{ posición } t\text{)} \\
score_{AKE,j} &\in \mathbb{R}^+ \quad \forall j \in J_{reg} \quad \text{(Puntuación AKE del ULD } j\text{)} \\
score_{PMC\_PAG,j} &\in \mathbb{R}^+ \quad \forall j \in J_{reg} \quad \text{(Puntuación PMC/PAG del ULD } j\text{)} \\
shortfall_j &\in \mathbb{R}^+ \quad \forall j \in J_{reg} \quad \text{(Subutilización del ULD } j\text{)} \\
sep_{p,j} &\in \{0,1\} \quad \forall p \in P, \forall j \in J \quad \text{(Penalización de separación del grupo } p \text{ en ULD } j\text{)} \\
ZFW\_index &\in \mathbb{R} \quad \text{(Índice Zero Fuel Weight)} \\
MAC &\in \mathbb{R} \quad \text{(Mean Aerodynamic Chord)}
\end{align}

\section{Función Objetivo Multi-objetivo}

\subsection{Objetivo 1: Optimización MAC (Prioridad 6)}
\begin{align}
\min \quad & MAC \\
\text{s.t.} \quad & MAC = \frac{\left( \frac{C \cdot (ZFW\_index - K)}{ZFW} + reference\_arm - lemac \right)}{mac\_formula / 100} \\
& ZFW\_index = DOI + INDEX\_PAX + \sum_{k=1}^{4} \left( \sum_{i \in I} \sum_{j \in J} \sum_{t \in T_{Ck}} w_{ijt} \cdot \Delta_{Ck} + \sum_{j \in J_{BAX} \cup J_{BUP} \cup J_T} \sum_{t \in T_{Ck}} w_j \cdot f_{jt} \cdot \Delta_{Ck} \right)
\end{align}

\subsection{Objetivo 2: Preferencia de Volumen (Prioridad 5)}
\begin{align}
\min \quad & \sum_{j \in J_{reg}} (score_{AKE,j} + score_{PMC\_PAG,j}) \\
\text{s.t.} \quad & score_{AKE,j} = \sum_{i \in I: v_i < threshold\_volume} p_{ij} \cdot (threshold\_volume - v_i) \cdot \mathbf{1}_{AKE}(j) \quad \forall j \in J_{reg} \\
& score_{PMC\_PAG,j} = \sum_{i \in I: v_i < threshold\_volume} p_{ij} \cdot (threshold\_volume - v_i) \cdot \mathbf{1}_{PMC\_PAG}(j) \quad \forall j \in J_{reg}
\end{align}

\subsection{Objetivo 3: Minimizar ULDs (Prioridad 4)}
\begin{equation}
\min \quad \sum_{j \in J} u_j
\end{equation}

\subsection{Objetivo 4: Minimizar Subutilización (Prioridad 3)}
\begin{align}
\min \quad & \sum_{j \in J_{reg}} shortfall_j \\
\text{s.t.} \quad & shortfall_j \geq min\_load\_factor - \frac{\sum_{i \in I} v_i \cdot p_{ij}}{V_j} \quad \forall j \in J_{reg} \\
& shortfall_j \geq 0 \quad \forall j \in J_{reg}
\end{align}

\subsection{Objetivo 5: Minimizar Separación (Prioridad 2)}
\begin{align}
\min \quad & \sum_{p \in P} \sum_{j \in J} sep_{p,j} \\
\text{s.t.} \quad & \sum_{i \in I_p} p_{ij} \leq |I_p| \cdot sep_{p,j} \quad \forall p \in P, \forall j \in J
\end{align}

\subsection{Objetivo 6: Proximidad BAX (Prioridad 1)}
\begin{equation}
\min \quad \sum_{j \in J_{BAX}} \sum_{t \in T} proximity\_score(t) \cdot f_{jt}
\end{equation}

\section{Restricciones}

\subsection{Restricciones de Asignación de Items}

\subsubsection{P1: Restricción de Apertura de ULDs}
\begin{equation}
\sum_{j \in J_{reg}} u_j \geq number\_of\_opened\_uld \tag{P1}
\end{equation}

\subsubsection{P2: Restricción de Uso de ULD}
\begin{equation}
\sum_{i \in I} p_{ij} \geq u_j \quad \forall j \in J_{reg} \tag{P2}
\end{equation}

\subsubsection{P3: Restricción de Capacidad de Peso}
\begin{equation}
\sum_{i \in I} w_i \cdot p_{ij} \leq W_j \cdot u_j \quad \forall j \in J \tag{P3}
\end{equation}

\subsubsection{P4: Restricción de Capacidad Volumétrica}
\begin{equation}
\sum_{i \in I} v_i \cdot p_{ij} \leq V_j \cdot u_j \cdot loadfactor \quad \forall j \in J \tag{P4}
\end{equation}

\subsubsection{P5: Restricción de Asignación Única}
\begin{equation}
\sum_{j \in J} p_{ij} = 1 \quad \forall i \in I \tag{P5}
\end{equation}

\subsubsection{P9: Restricción de Prohibición en BAX/BUP/T}
\begin{equation}
p_{ij} = 0 \quad \forall i \in I, \forall j \in J_{BAX} \cup J_{BUP} \cup J_T \tag{P9}
\end{equation}

\subsection{Restricciones de Asignación de ULDs}

\subsubsection{P6: Restricción de Asignación de ULD}
\begin{equation}
\sum_{t \in T} f_{jt} = u_j \quad \forall j \in J \tag{P6}
\end{equation}

\subsubsection{P7: Restricción de Posición Única}
\begin{equation}
\sum_{j \in J} f_{jt} \leq 1 \quad \forall t \in T \tag{P7}
\end{equation}

\subsubsection{P8: Restricción de Asignación BAX/BUP/T}
\begin{equation}
\sum_{t \in T} f_{jt} = 1 \quad \forall j \in J_{BAX} \cup J_{BUP} \cup J_T \tag{P8}
\end{equation}

\subsubsection{BF1: Restricción de Posición Fija BAX}
\begin{equation}
f_{j,t_{actual}} = 1 \quad \forall j \in J_{BAX}, \quad \text{donde } t_{actual} = actual\_position_{BAX}(j) \tag{BF1}
\end{equation}

\textbf{Nota Implementación:} Las posiciones BAX se obtienen del atributo \texttt{j.actual\_position\_bax} de cada ULD, que proviene de los datos reales de \texttt{LoadLocationsSpotfire.csv}.

\subsection{Restricciones de Linealización}

\subsubsection{Relación entre Variables de Peso y Asignación (L1-L7)}
\begin{align}
w_{ijt} &\leq M \cdot p_{ij} \quad \forall i \in I, \forall j \in J, \forall t \in T \tag{L1} \\
w_{ijt} &\leq M \cdot f_{jt} \quad \forall i \in I, \forall j \in J, \forall t \in T \tag{L2} \\
w_{ijt} &\geq w_i - M \cdot (1 - z_{ijt}) \quad \forall i \in I, \forall j \in J, \forall t \in T \tag{L6} \\
w_{ijt} &\leq w_i \quad \forall i \in I, \forall j \in J, \forall t \in T \tag{L7}
\end{align}

\subsubsection{Relación entre Variables de Asignación (L3-L5)}
\begin{align}
z_{ijt} &\leq p_{ij} \quad \forall i \in I, \forall j \in J, \forall t \in T \tag{L3} \\
z_{ijt} &\leq f_{jt} \quad \forall i \in I, \forall j \in J, \forall t \in T \tag{L4} \\
z_{ijt} &\geq p_{ij} + f_{jt} - 1 \quad \forall i \in I, \forall j \in J, \forall t \in T \tag{L5}
\end{align}

\subsection{Restricciones de Posiciones}

\subsubsection{P10: Restricción de Posiciones Prohibidas}
\begin{equation}
\sum_{t \in T_{forbidden}(j)} f_{jt} = 0 \quad \forall j \in J
\end{equation}

\subsubsection{P11: Restricción de Posiciones Superpuestas}
\begin{equation}
f_{j_1,t_1} + f_{j_2,t_2} \leq 1 \quad \forall j_1, j_2 \in J: j_1 \neq j_2, \forall t_1 \in T, \forall t_2 \in T_{overlap}(t_1)
\end{equation}

\subsection{Restricciones de Peso}

\subsubsection{P12: Restricción de Peso por Posición}
\begin{align}
\sum_{i \in I} \sum_{j \in J} w_{ijt} &\leq W_{max}(t) \quad \forall t \in T \tag{P12a} \\
\sum_{j \in J_{BAX} \cup J_{BUP} \cup J_T} w_j \cdot f_{jt} &\leq W_{max}(t) \tag{P12b} \quad \forall t \in T
\end{align}

\subsubsection{P13-P18: Restricciones de Peso por Compartimento}
\begin{align}
\sum_{i \in I} \sum_{j \in J} \sum_{t \in T_{C1}} w_{ijt} + \sum_{j \in J_{BAX} \cup J_{BUP} \cup J_T} \sum_{t \in T_{C1}} w_j \cdot f_{jt} &\leq W_{max}^{C1} \\
\sum_{i \in I} \sum_{j \in J} \sum_{t \in T_{C2}} w_{ijt} + \sum_{j \in J_{BAX} \cup J_{BUP} \cup J_T} \sum_{t \in T_{C2}} w_j \cdot f_{jt} &\leq W_{max}^{C2} \\
\sum_{i \in I} \sum_{j \in J} \sum_{t \in T_{C3}} w_{ijt} + \sum_{j \in J_{BAX} \cup J_{BUP} \cup J_T} \sum_{t \in T_{C3}} w_j \cdot f_{jt} &\leq W_{max}^{C3} \\
\sum_{i \in I} \sum_{j \in J} \sum_{t \in T_{C4}} w_{ijt} + \sum_{j \in J_{BAX} \cup J_{BUP} \cup J_T} \sum_{t \in T_{C4}} w_j \cdot f_{jt} &\leq W_{max}^{C4} \\
\sum_{i \in I} \sum_{j \in J} \sum_{t \in T_{C1} \cup T_{C2}} w_{ijt} + \sum_{j \in J_{BAX} \cup J_{BUP} \cup J_T} \sum_{t \in T_{C1} \cup T_{C2}} w_j \cdot f_{jt} &\leq W_{max}^{C1C2} \\
\sum_{i \in I} \sum_{j \in J} \sum_{t \in T_{C3} \cup T_{C4}} w_{ijt} + \sum_{j \in J_{BAX} \cup J_{BUP} \cup J_T} \sum_{t \in T_{C3} \cup T_{C4}} w_j \cdot f_{jt} &\leq W_{max}^{C3C4}
\end{align}

\subsubsection{P19: Restricción de Peso Total}
\begin{equation}
\sum_{i \in I} \sum_{j \in J} \sum_{t \in T} w_{ijt} + \sum_{j \in J_{BAX} \cup J_{BUP} \cup J_T} \sum_{t \in T} w_j \cdot f_{jt} \leq MPL
\end{equation}

\subsection{Restricciones de Balance}

\subsubsection{P20-P21: Restricciones de Balance Lateral TOW}
\begin{align}
&\left( \sum_{i \in I} \sum_{j \in J} \sum_{t \in T_{left}} w_{ijt} + \sum_{j \in J_{BAX} \cup J_{BUP} \cup J_T} \sum_{t \in T_{left}} w_j \cdot f_{jt} \right) - \\
&\left( \sum_{i \in I} \sum_{j \in J} \sum_{t \in T_{right}} w_{ijt} + \sum_{j \in J_{BAX} \cup J_{BUP} \cup J_T} \sum_{t \in T_{right}} w_j \cdot f_{jt} \right) \leq \\
&a_{lat}^{TOW} \cdot (W_{total} + OEW + TOF) \cdot b_{lat}^{TOW}
\end{align}

\begin{align}
&\left( \sum_{i \in I} \sum_{j \in J} \sum_{t \in T_{right}} w_{ijt} + \sum_{j \in J_{BAX} \cup J_{BUP} \cup J_T} \sum_{t \in T_{right}} w_j \cdot f_{jt} \right) - \\
&\left( \sum_{i \in I} \sum_{j \in J} \sum_{t \in T_{left}} w_{ijt} + \sum_{j \in J_{BAX} \cup J_{BUP} \cup J_T} \sum_{t \in T_{left}} w_j \cdot f_{jt} \right) \leq \\
&a_{lat}^{TOW} \cdot (W_{total} + OEW + TOF) \cdot b_{lat}^{TOW}
\end{align}

\subsubsection{P22-P23: Restricciones de Balance Lateral LW}
\begin{align}
&\left( \sum_{i \in I} \sum_{j \in J} \sum_{t \in T_{left}} w_{ijt} + \sum_{j \in J_{BAX} \cup J_{BUP} \cup J_T} \sum_{t \in T_{left}} w_j \cdot f_{jt} \right) - \\
&\left( \sum_{i \in I} \sum_{j \in J} \sum_{t \in T_{right}} w_{ijt} + \sum_{j \in J_{BAX} \cup J_{BUP} \cup J_T} \sum_{t \in T_{right}} w_j \cdot f_{jt} \right) \leq \\
&a_{lat}^{LW} \cdot (W_{total} + OEW + TOF - TripF) \cdot b_{lat}^{LW}
\end{align}

\begin{align}
&\left( \sum_{i \in I} \sum_{j \in J} \sum_{t \in T_{right}} w_{ijt} + \sum_{j \in J_{BAX} \cup J_{BUP} \cup J_T} \sum_{t \in T_{right}} w_j \cdot f_{jt} \right) - \\
&\left( \sum_{i \in I} \sum_{j \in J} \sum_{t \in T_{left}} w_{ijt} + \sum_{j \in J_{BAX} \cup J_{BUP} \cup J_T} \sum_{t \in T_{left}} w_j \cdot f_{jt} \right) \leq \\
&a_{lat}^{LW} \cdot (W_{total} + OEW + TOF - TripF) \cdot b_{lat}^{LW}
\end{align}

\subsubsection{P24-P25: Restricciones de Envelope CG TOW}
\begin{align}
INDEX\_TOW\_fwd &\leq DOI + fuel\_index + INDEX\_PAX + \tag{P24} \\
&\quad \sum_{k=1}^{4} \left( \sum_{i \in I} \sum_{j \in J} \sum_{t \in T_{Ck}} w_{ijt} \cdot \Delta_{Ck} + \sum_{j \in J_{BAX} \cup J_{BUP} \cup J_T} \sum_{t \in T_{Ck}} w_j \cdot f_{jt} \cdot \Delta_{Ck} \right) \\
DOI + fuel\_index + INDEX\_PAX + \\
&\quad \sum_{k=1}^{4} \left( \sum_{i \in I} \sum_{j \in J} \sum_{t \in T_{Ck}} w_{ijt} \cdot \Delta_{Ck} + \sum_{j \in J_{BAX} \cup J_{BUP} \cup J_T} \sum_{t \in T_{Ck}} w_j \cdot f_{jt} \cdot \Delta_{Ck} \right) \leq INDEX\_TOW\_aft \tag{P25}
\end{align}

\subsubsection{P26-P27: Restricciones de Envelope CG ZFW}
\begin{align}
INDEX\_ZFW\_fwd &\leq DOI + INDEX\_PAX + \tag{P26} \\
&\quad \sum_{k=1}^{4} \left( \sum_{i \in I} \sum_{j \in J} \sum_{t \in T_{Ck}} w_{ijt} \cdot \Delta_{Ck} + \sum_{j \in J_{BAX} \cup J_{BUP} \cup J_T} \sum_{t \in T_{Ck}} w_j \cdot f_{jt} \cdot \Delta_{Ck} \right) \\
DOI + INDEX\_PAX + \\
&\quad \sum_{k=1}^{4} \left( \sum_{i \in I} \sum_{j \in J} \sum_{t \in T_{Ck}} w_{ijt} \cdot \Delta_{Ck} + \sum_{j \in J_{BAX} \cup J_{BUP} \cup J_T} \sum_{t \in T_{Ck}} w_j \cdot f_{jt} \cdot \Delta_{Ck} \right) \leq INDEX\_ZFW\_aft \tag{P27}
\end{align}

\subsection{Restricciones de Manejo Especial}

\subsubsection{P28: Restricción de Manejo Especial (COL/CRT) en ULDs}
\begin{equation}
p_{i_1,j} + p_{i_2,j} \leq 1 \quad \forall j \in J_{reg}, \forall i_1, i_2 \in I: i_1 \neq i_2, COL_{i_1} = 1, CRT_{i_2} = 1
\end{equation}

\subsubsection{P29-P30: Restricciones de Manejo Especial COL/CRT por Compartimento}
Para aeronaves 772 y 77W:
\begin{align}
\sum_{j \in J_{COL}} f_{jt} + \sum_{j \in J_{CRT}} f_{jt} &\leq 1 \quad \forall t \in T_{C1} \cup T_{C2} \tag{P29a} \\
\sum_{j \in J_{COL}} f_{jt} + \sum_{j \in J_{CRT}} f_{jt} &\leq 1 \quad \forall t \in T_{C3} \cup T_{C4}
\end{align}

Para aeronaves 789 y 781:
\begin{equation}
\sum_{j \in J_{COL}} f_{jt} + \sum_{j \in J_{CRT}} f_{jt} = 0 \quad \forall t \in T_{C3} \cup T_{C4} \tag{P30}
\end{equation}

\section{Resultados del Modelo}

\subsection{Estadísticas de Rendimiento}
\begin{table}[H]
\centering
\begin{tabular}{@{}lcc@{}}
\toprule
Métrica & Valor & Descripción \\
\midrule
Total Vuelos Ejecutados & 241 & Vuelos procesados \\
Vuelos Válidos & 161 & Soluciones encontradas \\
Vuelos Infactibles & 80 & Sin solución \\
Tasa de Éxito & 66.8\% & Porcentaje de éxito \\
\bottomrule
\end{tabular}
\caption{Rendimiento del Modelo BAX Fixed}
\end{table}

\subsection{Análisis de Infeasibilidad}
El modelo BAX Fixed presenta la tasa de infactibilidad más alta (33.2\%), lo que indica:
\begin{itemize}
    \item Restricciones BAX demasiado estrictas
    \item Limitaciones en la flexibilidad de asignación
    \item Conflictos entre posiciones fijas y optimización
    \item Dificultades para encontrar soluciones factibles
\end{itemize}

\section{Ventajas y Desventajas}

\subsection{Ventajas}
\begin{itemize}
    \item Modelo integrado 1D-BPP + Weight \& Balance
    \item Simplicidad en la implementación de BAX
    \item Cumplimiento garantizado de restricciones BAX
    \item Alineación con operaciones reales de KLM
    \item Optimización multi-objetivo completa
\end{itemize}

\subsection{Desventajas}
\begin{itemize}
    \item Alta tasa de infactibilidad (33.2\%)
    \item Falta de flexibilidad en asignaciones BAX
    \item Posibles conflictos con optimización global
    \item Limitaciones en casos con múltiples BAX
    \item Complejidad computacional elevada
\end{itemize}

\section{Análisis Comparativo}

\subsection{Comparación con Otros Modelos}
\begin{table}[H]
\centering
\begin{tabular}{@{}lccc@{}}
\toprule
Modelo & Tasa de Éxito & Infactibilidad & Ranking \\
\midrule
Venezian & 89.8\% & 10.2\% & 1 \\
KLM\_Optimized & 84.6\% & 15.4\% & 2 \\
Baseline & 77.0\% & 23.0\% & 3 \\
Puttaert & 72.1\% & 27.9\% & 4 \\
\textbf{BAX Fixed} & \textbf{66.8\%} & \textbf{33.2\%} & \textbf{5} \\
\bottomrule
\end{tabular}
\caption{Comparación de Rendimiento entre Modelos}
\end{table}

\section{Conclusiones}

El modelo BAX Fixed representa un enfoque conservador que prioriza el cumplimiento de restricciones operacionales específicas sobre la optimización global. Su alta tasa de infactibilidad (33.2\%) sugiere que las restricciones fijas para ULDs BAX pueden ser demasiado restrictivas para muchos casos reales.

Aunque el modelo garantiza el cumplimiento de las restricciones BAX, su rendimiento general es el más bajo entre todos los modelos evaluados. Esto indica la necesidad de encontrar un equilibrio entre restricciones operacionales y flexibilidad de optimización.

Para mejorar el rendimiento del modelo, se recomienda:
\begin{itemize}
    \item Implementar restricciones BAX más flexibles
    \item Permitir múltiples posiciones posibles para BAX
    \item Optimizar el orden de asignación de ULDs
    \item Considerar restricciones suaves en lugar de fijas
    \item Implementar estrategias de relajación
\end{itemize}

\end{document}
